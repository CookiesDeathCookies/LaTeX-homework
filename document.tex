% Этот шаблон документа разработан в 2014 году
% Данилом Фёдоровых (danil@fedorovykh.ru) 
% для использования в курсе 
% <<Документы и презентации в LaTeX>>, записанном НИУ ВШЭ
% для Coursera.org: http://coursera.org/course/latex .
% Исходная версия шаблона --- 
% https://www.writelatex.com/coursera/latex/1.2

\documentclass[a4paper,10pt]{article} % добавить leqno в [] для нумерации слева

%%% Работа с русским языком
\usepackage{cmap}					% поиск в PDF
\usepackage{mathtext} 				% русские буквы в формулах
\usepackage[T2A]{fontenc}			% кодировка
\usepackage[utf8]{inputenc}			% кодировка исходного текста
\usepackage[english,russian]{babel}	% локализация и переносы
\usepackage[top=2cm,bottom=2cm,bindingoffset=0cm]{geometry}

%%% Дополнительная работа с математикой
\usepackage{amsmath,amsfonts,amssymb,amsthm,mathtools} % AMS
\usepackage{icomma} % "Умная" запятая: $0,2$ --- число, $0, 2$ --- перечисление

%% Номера формул
\mathtoolsset{showonlyrefs=true} % Показывать номера только у тех формул, на которые есть \eqref{} в тексте.

%% Шрифты
\usepackage{euscript}	 % Шрифт Евклид
\usepackage{mathrsfs} % Красивый матшрифт

%% Свои команды
\DeclareMathOperator{\sgn}{\mathop{sgn}}

%% Перенос знаков в формулах (по Львовскому)
\newcommand*{\hm}[1]{#1\nobreak\discretionary{}
	{\hbox{$\mathsurround=0pt #1$}}{}}

%%% Заголовок
\author{\LaTeX{} в Вышке}
\title{1.2 Математика в \LaTeX}
\date{\today}

\begin{document}
	\maketitle
	
	\section{Tasks}
	\subsection{Task 1}
	
	Составим расширенную матрицу коэффициентов и выполним определенные действия для решения системы.
	
	\begin{multline*}
		\begin{bmatrix}
		1 & 0 & 0 & 0 & 1 & 0 & 1 & 1 & 0 \\
		1 & 0 & 1 & 0 & 0 & 0 & 1 & 0 & 1 \\
		0 & 1 & 0 & 1 & 1 & 1 & 1 & 0 & 1 \\
		1 & 0 & 1 & 0 & 1 & 0 & 0 & 1 & 0 \\
		1 & 1 & 0 & 0 & 1 & 1 & 0 & 0 & 0 \\
		0 & 1 & 0 & 1 & 1 & 0 & 1 & 0 & 1 \\
		0 & 1 & 0 & 1 & 1 & 1 & 0 & 0 & 1 \\
		0 & 0 & 0 & 0 & 1 & 1 & 0 & 0 & 0 \\
		\end{bmatrix}
		\begin{align}
		(1) \oplus= (0)\\
		(3) \oplus= (0)\\
		(4) \oplus= (0)\\
		\end{align}
		\begin{bmatrix}
		1 & 0 & 0 & 0 & 1 & 0 & 1 & 1 & 0 \\
		0 & 0 & 1 & 0 & 1 & 0 & 0 & 1 & 1 \\
		0 & 1 & 0 & 1 & 1 & 1 & 1 & 0 & 1 \\
		0 & 0 & 1 & 0 & 0 & 0 & 1 & 0 & 0 \\
		0 & 1 & 0 & 0 & 0 & 1 & 1 & 1 & 0 \\
		0 & 1 & 0 & 1 & 1 & 0 & 1 & 0 & 1 \\
		0 & 1 & 0 & 1 & 1 & 1 & 0 & 0 & 1 \\
		0 & 0 & 0 & 0 & 1 & 1 & 0 & 0 & 0 \\
		\end{bmatrix}
		\begin{align}
		(4) \oplus= (2)\\
		(5) \oplus= (2)\\
		(6) \oplus= (2)\\
		\end{align}\\
		\begin{bmatrix}
		1 & 0 & 0 & 0 & 1 & 0 & 1 & 1 & 0 \\
		0 & 0 & 1 & 0 & 1 & 0 & 0 & 1 & 1 \\
		0 & 1 & 0 & 1 & 1 & 1 & 1 & 0 & 1 \\
		0 & 0 & 1 & 0 & 0 & 0 & 1 & 0 & 0 \\
		0 & 0 & 0 & 1 & 1 & 0 & 0 & 1 & 1 \\
		0 & 0 & 0 & 0 & 0 & 1 & 0 & 0 & 0 \\
		0 & 0 & 0 & 0 & 0 & 0 & 1 & 0 & 0 \\
		0 & 0 & 0 & 0 & 1 & 1 & 0 & 0 & 0 \\
		\end{bmatrix}
		\begin{align}
		(3) \oplus= (1)\\
		\end{align}
		\begin{bmatrix}
		1 & 0 & 0 & 0 & 1 & 0 & 1 & 1 & 0 \\
		0 & 0 & 1 & 0 & 1 & 0 & 0 & 1 & 1 \\
		0 & 1 & 0 & 1 & 1 & 1 & 1 & 0 & 1 \\
		0 & 0 & 0 & 0 & 1 & 0 & 1 & 1 & 1 \\
		0 & 0 & 0 & 1 & 1 & 0 & 0 & 1 & 1 \\
		0 & 0 & 0 & 0 & 0 & 1 & 0 & 0 & 0 \\
		0 & 0 & 0 & 0 & 0 & 0 & 1 & 0 & 0 \\
		0 & 0 & 0 & 0 & 1 & 1 & 0 & 0 & 0 \\
		\end{bmatrix}
		\begin{align}
		(2) \oplus= (4)\\
		\end{align}\\
		\begin{bmatrix}
		1 & 0 & 0 & 0 & 1 & 0 & 1 & 1 & 0 \\
		0 & 0 & 1 & 0 & 1 & 0 & 0 & 1 & 1 \\
		0 & 1 & 0 & 0 & 0 & 1 & 1 & 1 & 0 \\
		0 & 0 & 0 & 0 & 1 & 0 & 1 & 1 & 1 \\
		0 & 0 & 0 & 1 & 1 & 0 & 0 & 1 & 1 \\
		0 & 0 & 0 & 0 & 0 & 1 & 0 & 0 & 0 \\
		0 & 0 & 0 & 0 & 0 & 0 & 1 & 0 & 0 \\
		0 & 0 & 0 & 0 & 1 & 1 & 0 & 0 & 0 \\
		\end{bmatrix}
		\begin{align}
		(0) \oplus= (3)\\
		(1) \oplus= (3)\\
		(4) \oplus= (3)\\
		(7) \oplus= (3)\\
		\end{align}
		\begin{bmatrix}
		1 & 0 & 0 & 0 & 0 & 0 & 0 & 0 & 1 \\
		0 & 0 & 1 & 0 & 0 & 0 & 1 & 0 & 0 \\
		0 & 1 & 0 & 0 & 0 & 1 & 1 & 1 & 0 \\
		0 & 0 & 0 & 0 & 1 & 0 & 1 & 1 & 1 \\
		0 & 0 & 0 & 1 & 0 & 0 & 1 & 0 & 0 \\
		0 & 0 & 0 & 0 & 0 & 1 & 0 & 0 & 0 \\
		0 & 0 & 0 & 0 & 0 & 0 & 1 & 0 & 0 \\
		0 & 0 & 0 & 0 & 0 & 1 & 1 & 1 & 1 \\
		\end{bmatrix}
		\begin{align}
		(2) \oplus= (5)\\
		(7) \oplus= (5)\\
		\end{align}\\
		\begin{bmatrix}
		1 & 0 & 0 & 0 & 0 & 0 & 0 & 0 & 1 \\
		0 & 0 & 1 & 0 & 0 & 0 & 1 & 0 & 0 \\
		0 & 1 & 0 & 0 & 0 & 0 & 1 & 1 & 0 \\
		0 & 0 & 0 & 0 & 1 & 0 & 1 & 1 & 1 \\
		0 & 0 & 0 & 1 & 0 & 0 & 1 & 0 & 0 \\
		0 & 0 & 0 & 0 & 0 & 1 & 0 & 0 & 0 \\
		0 & 0 & 0 & 0 & 0 & 0 & 1 & 0 & 0 \\
		0 & 0 & 0 & 0 & 0 & 0 & 1 & 1 & 1 \\
		\end{bmatrix}
		\begin{align}
		(1) \oplus= (6)\\
		(2) \oplus= (6)\\
		(3) \oplus= (6)\\
		(4) \oplus= (6)\\
		(7) \oplus= (6)\\
		\end{align}
		\begin{bmatrix}
		1 & 0 & 0 & 0 & 0 & 0 & 0 & 0 & 1 \\
		0 & 0 & 1 & 0 & 0 & 0 & 0 & 0 & 0 \\
		0 & 1 & 0 & 0 & 0 & 0 & 0 & 1 & 0 \\
		0 & 0 & 0 & 0 & 1 & 0 & 0 & 1 & 1 \\
		0 & 0 & 0 & 1 & 0 & 0 & 0 & 0 & 0 \\
		0 & 0 & 0 & 0 & 0 & 1 & 0 & 0 & 0 \\
		0 & 0 & 0 & 0 & 0 & 0 & 1 & 0 & 0 \\
		0 & 0 & 0 & 0 & 0 & 0 & 0 & 1 & 1 \\
		\end{bmatrix}
		\begin{align}
		(2) \oplus= (7)\\
		(3) \oplus= (7)\\
		\end{align}	
	\end{multline*}
	
	В итоге получаем матрицу:
	\begin{bmatrix}
	1 & 0 & 0 & 0 & 0 & 0 & 0 & 0 & 1 \\
	0 & 0 & 1 & 0 & 0 & 0 & 0 & 0 & 0 \\
	0 & 1 & 0 & 0 & 0 & 0 & 0 & 0 & 1 \\
	0 & 0 & 0 & 0 & 1 & 0 & 0 & 0 & 0 \\
	0 & 0 & 0 & 1 & 0 & 0 & 0 & 0 & 0 \\
	0 & 0 & 0 & 0 & 0 & 1 & 0 & 0 & 0 \\
	0 & 0 & 0 & 0 & 0 & 0 & 1 & 0 & 0 \\
	0 & 0 & 0 & 0 & 0 & 0 & 0 & 1 & 1 \\
	\end{bmatrix}


	В описаниях преобразований строки пронумерованы сверху вниз $(0),
	(1), …, (6), (7)$, а выражение $(i) \oplus= (j)$ обозначает  
	«заменить все числа в строке (i) на их сумму по модулю 2 с соответствующими числами строки (j)».
	
	Получаем решение: X_{7} = 1, X_{6} = 1, X_{5} = 0, X_{4} = 0, X_{3} = 0, X_{2} = 0, X_{1} = 0, X_{0} = 1.

	Десятичный номер функции равен 2^7 + 2^6 + 2^0 = 193. 
	
	Таблица истинности для данной функции:
	
	\begin{tabular}{|c|c|c|c|c|c|c|c|c|}
		\hline 
		A & 0 & 0 & 0 & 0 & 1 & 1 & 1 & 1 \\ 
		\hline 
		B & 0 & 0 & 1 & 1 & 0 & 0 & 1 & 1 \\ 
		\hline 
		C & 0 & 1 & 0 & 1 & 0 & 1 & 0 & 1 \\ 
		\hline 
		F & 1 & 1 & 0 & 0 & 0 & 0 & 0 & 1 \\ 
		\hline 
	\end{tabular} 
	
	\subsection{Task 2}
	
	Представим таблицы истинности функции F в виде карты Карно:
	
	\begin{tabular}{|c|c|c|c|c|c|}
		\hline 
		F & 00 & 01 & 10 & 11 & AB \\ 
		\hline 
		0 & 1 & 0 & 0 & 0 & \\ 
		\hline 
		1 & 1 & 0 & 0 & 1 & \\ 
		\hline 
		C &  &  &  &  &  \\ 
		\hline
	\end{tabular} 
\end{document}