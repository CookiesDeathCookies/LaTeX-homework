% Этот шаблон документа разработан в 2014 году
% Данилом Фёдоровых (danil@fedorovykh.ru) 
% для использования в курсе 
% <<Документы и презентации в LaTeX>>, записанном НИУ ВШЭ
% для Coursera.org: http://coursera.org/course/latex .
% Исходная версия шаблона --- 
% https://www.writelatex.com/coursera/latex/1.2

\documentclass[a4paper,10pt]{article} % добавить leqno в [] для нумерации слева

%%% Работа с русским языком
\usepackage{cmap}					% поиск в PDF
\usepackage{mathtext} 				% русские буквы в формулах
\usepackage[T2A]{fontenc}			% кодировка
\usepackage[utf8]{inputenc}			% кодировка исходного текста
\usepackage[english,russian]{babel}	% локализация и переносы
\usepackage[top=2cm,bottom=2cm,bindingoffset=0cm]{geometry}

%%% Дополнительная работа с математикой
\usepackage{amsmath,amsfonts,amssymb,amsthm,mathtools} % AMS
\usepackage{centernot}
\usepackage{icomma} % "Умная" запятая: $0,2$ --- число, $0, 2$ --- перечисление

%% Номера формул
\mathtoolsset{showonlyrefs=true} % Показывать номера только у тех формул, на которые есть \eqref{} в тексте.

%% Шрифты
\usepackage{euscript}	 % Шрифт Евклид
\usepackage{mathrsfs} % Красивый матшрифт

%% Свои команды
\DeclareMathOperator{\sgn}{\mathop{sgn}}

%% Перенос знаков в формулах (по Львовскому)
\newcommand*{\hm}[1]{#1\nobreak\discretionary{}
	{\hbox{$\mathsurround=0pt #1$}}{}}

%%% Заголовок
\author{\LaTeX{} в Вышке}
\title{1.2 Математика в \LaTeX}
\date{\today}

\begin{document}
	\maketitle
	
	\section{Решение системы уравнений и составление таблицы истинности функции}
	
	Составим расширенную матрицу коэффициентов и выполним определенные действия для решения системы.
	\begin{multline}
		\begin{bmatrix}
		1 & 0 & 0 & 0 & 1 & 0 & 1 & 1 & 0 \\
		1 & 0 & 1 & 0 & 0 & 0 & 1 & 0 & 1 \\
		0 & 1 & 0 & 1 & 1 & 1 & 1 & 0 & 1 \\
		1 & 0 & 1 & 0 & 1 & 0 & 0 & 1 & 0 \\
		1 & 1 & 0 & 0 & 1 & 1 & 0 & 0 & 0 \\
		0 & 1 & 0 & 1 & 1 & 0 & 1 & 0 & 1 \\
		0 & 1 & 0 & 1 & 1 & 1 & 0 & 0 & 1 \\
		0 & 0 & 0 & 0 & 1 & 1 & 0 & 0 & 0 \\
		\end{bmatrix}
		\begin{align}
		(1) \oplus= (0)\\
		(3) \oplus= (0)\\
		(4) \oplus= (0)\\
		\end{align}
		\begin{bmatrix}
		1 & 0 & 0 & 0 & 1 & 0 & 1 & 1 & 0 \\
		0 & 0 & 1 & 0 & 1 & 0 & 0 & 1 & 1 \\
		0 & 1 & 0 & 1 & 1 & 1 & 1 & 0 & 1 \\
		0 & 0 & 1 & 0 & 0 & 0 & 1 & 0 & 0 \\
		0 & 1 & 0 & 0 & 0 & 1 & 1 & 1 & 0 \\
		0 & 1 & 0 & 1 & 1 & 0 & 1 & 0 & 1 \\
		0 & 1 & 0 & 1 & 1 & 1 & 0 & 0 & 1 \\
		0 & 0 & 0 & 0 & 1 & 1 & 0 & 0 & 0 \\
		\end{bmatrix}
		\begin{align}
		(4) \oplus= (2)\\
		(5) \oplus= (2)\\
		(6) \oplus= (2)\\
		\end{align}\\
		\begin{bmatrix}
		1 & 0 & 0 & 0 & 1 & 0 & 1 & 1 & 0 \\
		0 & 0 & 1 & 0 & 1 & 0 & 0 & 1 & 1 \\
		0 & 1 & 0 & 1 & 1 & 1 & 1 & 0 & 1 \\
		0 & 0 & 1 & 0 & 0 & 0 & 1 & 0 & 0 \\
		0 & 0 & 0 & 1 & 1 & 0 & 0 & 1 & 1 \\
		0 & 0 & 0 & 0 & 0 & 1 & 0 & 0 & 0 \\
		0 & 0 & 0 & 0 & 0 & 0 & 1 & 0 & 0 \\
		0 & 0 & 0 & 0 & 1 & 1 & 0 & 0 & 0 \\
		\end{bmatrix}
		\begin{align}
		(3) \oplus= (1)\\
		\end{align}
		\begin{bmatrix}
		1 & 0 & 0 & 0 & 1 & 0 & 1 & 1 & 0 \\
		0 & 0 & 1 & 0 & 1 & 0 & 0 & 1 & 1 \\
		0 & 1 & 0 & 1 & 1 & 1 & 1 & 0 & 1 \\
		0 & 0 & 0 & 0 & 1 & 0 & 1 & 1 & 1 \\
		0 & 0 & 0 & 1 & 1 & 0 & 0 & 1 & 1 \\
		0 & 0 & 0 & 0 & 0 & 1 & 0 & 0 & 0 \\
		0 & 0 & 0 & 0 & 0 & 0 & 1 & 0 & 0 \\
		0 & 0 & 0 & 0 & 1 & 1 & 0 & 0 & 0 \\
		\end{bmatrix}
		\begin{align}
		(2) \oplus= (4)\\
		\end{align}\\
		\begin{bmatrix}
		1 & 0 & 0 & 0 & 1 & 0 & 1 & 1 & 0 \\
		0 & 0 & 1 & 0 & 1 & 0 & 0 & 1 & 1 \\
		0 & 1 & 0 & 0 & 0 & 1 & 1 & 1 & 0 \\
		0 & 0 & 0 & 0 & 1 & 0 & 1 & 1 & 1 \\
		0 & 0 & 0 & 1 & 1 & 0 & 0 & 1 & 1 \\
		0 & 0 & 0 & 0 & 0 & 1 & 0 & 0 & 0 \\
		0 & 0 & 0 & 0 & 0 & 0 & 1 & 0 & 0 \\
		0 & 0 & 0 & 0 & 1 & 1 & 0 & 0 & 0 \\
		\end{bmatrix}
		\begin{align}
		(0) \oplus= (3)\\
		(1) \oplus= (3)\\
		(4) \oplus= (3)\\
		(7) \oplus= (3)\\
		\end{align}
		\begin{bmatrix}
		1 & 0 & 0 & 0 & 0 & 0 & 0 & 0 & 1 \\
		0 & 0 & 1 & 0 & 0 & 0 & 1 & 0 & 0 \\
		0 & 1 & 0 & 0 & 0 & 1 & 1 & 1 & 0 \\
		0 & 0 & 0 & 0 & 1 & 0 & 1 & 1 & 1 \\
		0 & 0 & 0 & 1 & 0 & 0 & 1 & 0 & 0 \\
		0 & 0 & 0 & 0 & 0 & 1 & 0 & 0 & 0 \\
		0 & 0 & 0 & 0 & 0 & 0 & 1 & 0 & 0 \\
		0 & 0 & 0 & 0 & 0 & 1 & 1 & 1 & 1 \\
		\end{bmatrix}
		\begin{align}
		(2) \oplus= (5)\\
		(7) \oplus= (5)\\
		\end{align}\\
		\begin{bmatrix}
		1 & 0 & 0 & 0 & 0 & 0 & 0 & 0 & 1 \\
		0 & 0 & 1 & 0 & 0 & 0 & 1 & 0 & 0 \\
		0 & 1 & 0 & 0 & 0 & 0 & 1 & 1 & 0 \\
		0 & 0 & 0 & 0 & 1 & 0 & 1 & 1 & 1 \\
		0 & 0 & 0 & 1 & 0 & 0 & 1 & 0 & 0 \\
		0 & 0 & 0 & 0 & 0 & 1 & 0 & 0 & 0 \\
		0 & 0 & 0 & 0 & 0 & 0 & 1 & 0 & 0 \\
		0 & 0 & 0 & 0 & 0 & 0 & 1 & 1 & 1 \\
		\end{bmatrix}
		\begin{align}
		(1) \oplus= (6)\\
		(2) \oplus= (6)\\
		(3) \oplus= (6)\\
		(4) \oplus= (6)\\
		(7) \oplus= (6)\\
		\end{align}
		\begin{bmatrix}
		1 & 0 & 0 & 0 & 0 & 0 & 0 & 0 & 1 \\
		0 & 0 & 1 & 0 & 0 & 0 & 0 & 0 & 0 \\
		0 & 1 & 0 & 0 & 0 & 0 & 0 & 1 & 0 \\
		0 & 0 & 0 & 0 & 1 & 0 & 0 & 1 & 1 \\
		0 & 0 & 0 & 1 & 0 & 0 & 0 & 0 & 0 \\
		0 & 0 & 0 & 0 & 0 & 1 & 0 & 0 & 0 \\
		0 & 0 & 0 & 0 & 0 & 0 & 1 & 0 & 0 \\
		0 & 0 & 0 & 0 & 0 & 0 & 0 & 1 & 1 \\
		\end{bmatrix}
		\begin{align}
		(2) \oplus= (7)\\
		(3) \oplus= (7)\\
		\end{align}	
	\end{multline}
	
	\noindent В итоге получаем матрицу:
	\begin{bmatrix}
	1 & 0 & 0 & 0 & 0 & 0 & 0 & 0 & 1 \\
	0 & 0 & 1 & 0 & 0 & 0 & 0 & 0 & 0 \\
	0 & 1 & 0 & 0 & 0 & 0 & 0 & 0 & 1 \\
	0 & 0 & 0 & 0 & 1 & 0 & 0 & 0 & 0 \\
	0 & 0 & 0 & 1 & 0 & 0 & 0 & 0 & 0 \\
	0 & 0 & 0 & 0 & 0 & 1 & 0 & 0 & 0 \\
	0 & 0 & 0 & 0 & 0 & 0 & 1 & 0 & 0 \\
	0 & 0 & 0 & 0 & 0 & 0 & 0 & 1 & 1 \\
	\end{bmatrix} \\
	
	\noindent В описаниях преобразований строки пронумерованы сверху вниз $(0),
	(1), …, (6), (7)$, а выражение $(i) \oplus= (j)$ обозначает  
	«заменить все числа в строке (i) на их сумму по модулю 2 с соответствующими числами строки (j)». \\

	\noindent Получаем решение:$ X_{7} = 1, X_{6} = 1, X_{5} = 0, X_{4} = 0, X_{3} = 0, X_{2} = 0, X_{1} = 0, X_{0} = 1$. Десятичный номер функции равен $2^7 + 2^6 + 2^0 = 193$. \\ 
	
	\noindent Таблица истинности для данной функции:\\
	
	\begin{tabular}{|c|c|c|c|c|c|c|c|c|}
		\hline 
		A & 0 & 0 & 0 & 0 & 1 & 1 & 1 & 1 \\ 
		\hline 
		B & 0 & 0 & 1 & 1 & 0 & 0 & 1 & 1 \\ 
		\hline 
		C & 0 & 1 & 0 & 1 & 0 & 1 & 0 & 1 \\ 
		\hline 
		F & 1 & 1 & 0 & 0 & 0 & 0 & 0 & 1 \\ 
		\hline 
	\end{tabular} 
	
	\section{Карта Карно}
	
	Представим таблицу истинности функции F в виде карты Карно: \\
	
	\begin{tabular}{|c|c|c|c|c|c|}
		\hline 
		& 0 & 0 & 1 & 1 & A \\ 
		\hline 
		& 0 & 1 & 1 & 0 & B \\ 
		\hline 
		0 & 1 & 0 & 0 & 0 &  \\ 
		\hline 
		1 & 1 & 0 & 1 & 0 &  \\ 
		\hline 
		C &  &  &  &  &  \\ 
		\hline 
	\end{tabular} 

	\section{Дизъюнктивные разложения Шеннона}
	
	\begin{equation}
		\left(
		\begin{smallmatrix}
			A & 0 & 0 & 0 & 0 & 1 & 1 & 1 & 1 \\
			B & 0 & 0 & 1 & 1 & 0 & 0 & 1 & 1 \\
			C & 0 & 1 & 0 & 1 & 0 & 1 & 0 & 1 \\
			F & 1 & 1 & 0 & 0 & 0 & 0 & 0 & 1
		\end{smallmatrix}
		\right)
		= 
		\overline{A} \cdot 
		\left(
		\begin{smallmatrix}
			B & 0 & 0 & 1 & 1 \\
			C & 0 & 1 & 0 & 1 \\
			F & 1 & 1 & 0 & 0
		\end{smallmatrix} 
		\right)
		+ 
		A \cdot
		\left(
		\begin{smallmatrix}
			B & 0 & 0 & 1 & 1 \\
			C & 0 & 1 & 0 & 1 \\
			F & 0 & 0 & 0 & 1
		\end{smallmatrix}
		\right)
		= \overline{A} \cdot \overline{B} + A \cdot B \cdot C
	\end{equation}
	
	\begin{equation}
		\left(
		\begin{smallmatrix}
			A & 0 & 0 & 0 & 0 & 1 & 1 & 1 & 1 \\
			B & 0 & 0 & 1 & 1 & 0 & 0 & 1 & 1 \\
			C & 0 & 1 & 0 & 1 & 0 & 1 & 0 & 1 \\
			F & 1 & 1 & 0 & 0 & 0 & 0 & 0 & 1
		\end{smallmatrix}
		\right)
		= \overline{B} \cdot
		\left(
		\begin{smallmatrix}
			A & 0 & 0 & 1 & 1 \\
			C & 0 & 1 & 0 & 1 \\
			F & 1 & 1 & 0 & 0 
		\end{smallmatrix}
		\right)
		+ B \cdot
		\left(
		\begin{smallmatrix}
			A & 0 & 0 & 1 & 1 \\
			C & 0 & 1 & 0 & 1 \\
			F & 0 & 0 & 0 & 1
		\end{smallmatrix}
		\right)
		= \overline{A} \cdot \overline{B} + A \cdot B \cdot C
	\end{equation}

	\begin{equation}
		\left(
		\begin{smallmatrix}
			A & 0 & 0 & 0 & 0 & 1 & 1 & 1 & 1 \\
			B & 0 & 0 & 1 & 1 & 0 & 0 & 1 & 1 \\
			C & 0 & 1 & 0 & 1 & 0 & 1 & 0 & 1 \\
			F & 1 & 1 & 0 & 0 & 0 & 0 & 0 & 1
		\end{smallmatrix}
		\right)
		= \overline{C} \cdot 
		\left(
		\begin{smallmatrix}
			A & 0 & 0 & 1 & 1 \\
			B & 0 & 1 & 0 & 1 \\
			F & 1 & 0 & 0 & 0
		\end{smallmatrix}
		\right)
		+ C \cdot
		\left(
		\begin{smallmatrix}
			A & 0 & 0 & 1 & 1 \\
			B & 0 & 1 & 0 & 1 \\
			F & 1 & 0 & 0 & 1
		\end{smallmatrix}
		\right)
		= \overline{A} \cdot \overline{B} \cdot \overline{C} + C \cdot (A \equiv B) 
	\end{equation}
	
	\section{Совершенная дизъюнктивная нормальная форма}
	
	Совершенная дизъюнктивная нормальная форма: 
	
	\begin{equation}
		F(A, B, C) = \overline{A} \cdot \overline{B} \cdot \overline{C} +
		\overline{A} \cdot \overline{B} \cdot C + 
		A \cdot B \cdot C 
	\end{equation}
	
	\section{Минимальные дизъюнктивные нормальные формы}
	
	Воспользовавшись картой Карно, получим минимальную дизъюнктивную форму (она единственна, потому что единицы можно сгруппировать лишь одним способом):
	\begin{equation}
		F(A, B, C) = \overline{A} \cdot \overline{B} + A \cdot B \cdot C 
	\end{equation}

	\section{Новые представления функции}
	
	Из дизъюнктивных разложений, используя ортогональность, получим новые представления функции.
	
	\begin{equation}
		F(A, B, C) = \overline{A} \cdot \overline{B} \oplus A \cdot B \cdot C
	\end{equation}
	\begin{equation}
		F(A, B, C) = \overline{A} \cdot \overline{B} \cdot \overline{C} 
		\oplus C \cdot (A \equiv B) 
	\end{equation}
	\begin{equation}
		F(A, B, C) = \overline{A} \cdot \overline{B} \cdot \overline{C} \oplus
		\overline{A} \cdot \overline{B} \cdot C \oplus
		A \cdot B \cdot C 
	\end{equation}
	
	\section{Конъюнктивные разложения Шеннона}
	
	\begin{equation}
		\left(
		\begin{smallmatrix}
			A & 0 & 0 & 0 & 0 & 1 & 1 & 1 & 1 \\
			B & 0 & 0 & 1 & 1 & 0 & 0 & 1 & 1 \\
			C & 0 & 1 & 0 & 1 & 0 & 1 & 0 & 1 \\
			F & 1 & 1 & 0 & 0 & 0 & 0 & 0 & 1
		\end{smallmatrix}
		\right)
		= \left(A + 
		\left(
		\begin{smallmatrix}
			B & 0 & 0 & 1 & 1 \\
			C & 0 & 1 & 0 & 1 \\
			F & 1 & 1 & 0 & 0
		\end{smallmatrix}
		\right)
		\right) \cdot \left( \overline{A} + 
		\left(
		\begin{smallmatrix}
			B & 0 & 0 & 1 & 1 \\
			C & 0 & 1 & 0 & 1 \\
			F & 0 & 0 & 0 & 1
		\end{smallmatrix}
		\right)
		\right) = (A + \overline{B}) \cdot (\overline{A} + B \cdot C)
	\end{equation}
	
	\begin{equation}
		\left(
		\begin{smallmatrix}
			A & 0 & 0 & 0 & 0 & 1 & 1 & 1 & 1 \\
			B & 0 & 0 & 1 & 1 & 0 & 0 & 1 & 1 \\
			C & 0 & 1 & 0 & 1 & 0 & 1 & 0 & 1 \\
			F & 1 & 1 & 0 & 0 & 0 & 0 & 0 & 1
		\end{smallmatrix}
		\right)
		= \left(B + 
		\left(
		\begin{smallmatrix}
			A & 0 & 0 & 1 & 1 \\
			C & 0 & 1 & 0 & 1 \\
			F & 1 & 1 & 0 & 0
		\end{smallmatrix}
		\right)
		\right) \cdot \left(\overline{B} + 
		\left(
		\begin{smallmatrix}
			A & 0 & 0 & 1 & 1 \\
			C & 0 & 1 & 0 & 1 \\
			F & 0 & 0 & 0 & 1
		\end{smallmatrix}  
		\right)
		\right) = (B + \overline{A}) \cdot (\overline{B} + A \cdot C)
	\end{equation}
	
	\begin{equation}
		\left(
		\begin{smallmatrix}
			A & 0 & 0 & 0 & 0 & 1 & 1 & 1 & 1 \\
			B & 0 & 0 & 1 & 1 & 0 & 0 & 1 & 1 \\
			C & 0 & 1 & 0 & 1 & 0 & 1 & 0 & 1 \\
			F & 1 & 1 & 0 & 0 & 0 & 0 & 0 & 1
		\end{smallmatrix}
		\right)
		= \left( C + 
		\left(
		\begin{smallmatrix}
			A & 0 & 0 & 1 & 1 \\
			B & 0 & 1 & 0 & 1 \\
			F & 1 & 0 & 0 & 0
		\end{smallmatrix}
		\right)
		\right) \cdot \left( \overline{C} +
		\left(
		\begin{smallmatrix}
			A & 0 & 0 & 1 & 1 \\
			B & 0 & 1 & 0 & 1 \\
			F & 1 & 0 & 0 & 1 
		\end{smallmatrix}
		\right)
		\right) = (C + \overline{A} \cdot \overline{B}) \cdot
		 (\overline{C} + (A \equiv B)) 
	\end{equation}

	\section{Соверешенная конъюнктивная нормальная форма}
	
	Воспользовавшись двойственностью, получим СКНФ из СДНФ.
	
	\begin{equation}
		F(A, B, C) = (A + B + \overline{C}) \cdot 
		(A + \overline{B} + C) \cdot (A + \overline{B} + \overline{C}) \cdot
		(\overline{A} + B + C) \cdot (\overline{A} + B + \overline{C})
	\end{equation}
	
	\section{Минимальные конъюнктивные нормальные формы}
	По карте Карно мы можем увидеть, что нули можно сгруппировать двумя способами.
	Соответсвенно, получим две минимальные конъюнктивные нормальные формы.
	\begin{equation}
		F(A, B, C) = (A + \overline{B})
		 \cdot (\overline{B} + C) 
		 \cdot (\overline{A} + B)
	\end{equation}
	\begin{equation}
		F(A, B, C) = (A + \overline{B})
		\cdot (\overline{A} + C) 
		\cdot (\overline{A} + B)
	\end{equation}
	
	\section{Новые представления функции}
	Из конъюнктивных разложений, используя ортогональность, получим новые представления функции
	\begin{equation}
		F(A, B, C) = (A + \overline{B}) \equiv (\overline{A} + B \cdot C)
	\end{equation}
	\begin{equation}
		F(A, B, C) = (B + \overline{A}) \equiv (\overline{B} + A \cdot C)
	\end{equation}
	\begin{equation}
		F(A, B, C) = (C + \overline{A} \cdot \overline{B}) \equiv
		(\overline{C} + (A \equiv B)) 
	\end{equation}
	\begin{equation}
		F(A, B, C) = (A + B + \overline{C}) \equiv
		(A + \overline{B} + C) \equiv (A + \overline{B} + \overline{C}) \equiv
		(\overline{A} + B + C) \equiv (\overline{A} + B + \overline{C})
	\end{equation}
	
	\section{Производные функции}
	
	\begin{equation}
		F^{\prime}_{A} = 
		\left(
		\begin{smallmatrix}
			A & 0 & 0 & 0 & 0 & 1 & 1 & 1 & 1 \\
			B & 0 & 0 & 1 & 1 & 0 & 0 & 1 & 1 \\
			C & 0 & 1 & 0 & 1 & 0 & 1 & 0 & 1 \\
			F & 1 & 1 & 0 & 0 & 0 & 0 & 0 & 1
		\end{smallmatrix}
		\right)^{\prime}_{A} = 
		\left(
		\begin{smallmatrix}
			B & 0 & 0 & 1 & 1 \\
			C & 0 & 1 & 0 & 1 \\
			F & 1 & 1 & 0 & 0
		\end{smallmatrix} 
		\right) \oplus \left(
		\begin{smallmatrix}
			B & 0 & 0 & 1 & 1 \\
			C & 0 & 1 & 0 & 1 \\
			F & 0 & 0 & 0 & 1
		\end{smallmatrix}
		\right)
		= B \implies C
	\end{equation}
	
	\begin{equation}
		F^{\prime}_{B} = 
		\left(
		\begin{smallmatrix}
			A & 0 & 0 & 0 & 0 & 1 & 1 & 1 & 1 \\
			B & 0 & 0 & 1 & 1 & 0 & 0 & 1 & 1 \\
			C & 0 & 1 & 0 & 1 & 0 & 1 & 0 & 1 \\
			F & 1 & 1 & 0 & 0 & 0 & 0 & 0 & 1
		\end{smallmatrix}
		\right) ^ {\prime}_{B} = 
		\left(
		\begin{smallmatrix}
			A & 0 & 0 & 1 & 1 \\
			C & 0 & 1 & 0 & 1 \\
			F & 1 & 1 & 0 & 0
		\end{smallmatrix}
		\right) \oplus 
		\left( 
		\begin{smallmatrix}
			A & 0 & 0 & 1 & 1 \\
			C & 0 & 1 & 0 & 1 \\
			F & 0 & 0 & 0 & 1
		\end{smallmatrix}  
		\right) 
		= A \implies C
	\end{equation}

	\begin{equation}
		F^{\prime}_{C} = 
		\left(
		\begin{smallmatrix}
			A & 0 & 0 & 0 & 0 & 1 & 1 & 1 & 1 \\
			B & 0 & 0 & 1 & 1 & 0 & 0 & 1 & 1 \\
			C & 0 & 1 & 0 & 1 & 0 & 1 & 0 & 1 \\
			F & 1 & 1 & 0 & 0 & 0 & 0 & 0 & 1
		\end{smallmatrix}
		\right)^{\prime}_{C} =
		\left(
		\begin{smallmatrix}
			A & 0 & 0 & 1 & 1 \\
			B & 0 & 1 & 0 & 1 \\
			F & 1 & 0 & 0 & 0
		\end{smallmatrix}
		\right) \oplus
		\left(
		\begin{smallmatrix}
			A & 0 & 0 & 1 & 1 \\
			B & 0 & 1 & 0 & 1 \\
			F & 1 & 0 & 0 & 1 
		\end{smallmatrix}
		\right)
		= A \cdot B
	\end{equation}

	
	\section{Разложения Рида}
	Воспользовавшись формулами $F(x) = F(0) \oplus x \cdot F_{x}^{\prime}$ 
	и $F(x) = F(1) \oplus \overline{x} \cdot F_{x}^{\prime}$, получим следующие
	разложения:
	
	\begin{equation}
		F(A, B, C) = F(0, B, C) \oplus A \cdot F_{A}^{\prime}(A, B, C) =
		\overline{B} \oplus A \cdot (B \implies C) 
	\end{equation}
	\begin{equation}
		F(A, B, C) = F(1, B, C) \oplus \overline{A} \cdot F_{A}^{\prime}(A, B, C) =
		B \cdot C \oplus \overline{A} \cdot (B \implies C)
	\end{equation}
	\begin{equation}
		F(A, B, C) = F(A, 0, C) \oplus B \cdot F_{B}^{\prime}(A, B, C) =
		\overline{A} \oplus B \cdot (A \implies C)
	\end{equation}
	\begin{equation}
		F(A, B, C) = F(A, 1, C) \oplus \overline{B} \cdot F_{B}^{\prime}(A, B, C) =
		A \cdot C \oplus \overline{B} \cdot (A \implies C)
	\end{equation}
	\begin{equation}
		F(A, B, C) = F(A, B, 0) \oplus C \cdot F_{C}^{\prime}(A, B, C) =
		\overline{A} \cdot \overline{B} \oplus C \cdot (A \cdot B)
	\end{equation}
	\begin{equation}
		F(A, B, C) = F(A, B, 1) \oplus \overline{C} \cdot F_{C}^{\prime}(A, B, C) =
		(A \equiv B) \oplus \overline{C} \cdot (A \cdot B) 
	\end{equation}

	\section{Двойственные разложения Рида}
		
	Воспользуемся формулами $F(x) = F(0) \equiv (\overline{x} + \overline{F_{x}^{\prime}})$ и $F(x) = F(1) \equiv (x + \overline{F_{x}^{\prime}})$ получим двойственные разложения Рида.
	
	\begin{equation}
		F(A, B, C) = F(0, B, C) \equiv (\overline{A} + \overline{F_{A}^{\prime}})
		= \overline{B} \equiv (\overline{A} + (B \centernot\implies C)) 
	\end{equation}
	\begin{equation}
		F(A, B, C) = F(1, B, C) \equiv (A + \overline{F_{A}^{\prime}}) =
		B \cdot C \equiv (A + (B \centernot\implies C))
	\end{equation}
	\begin{equation}
		F(A, B, C) = F(A, 0, C) \equiv (\overline{B} + \overline{F_{B}^{\prime}})
		= \overline{A} \equiv (\overline{B} + (A \centernot\implies C))
	\end{equation}
	\begin{equation}
		F(A, B, C) = F(A, 1, C) \equiv (B + \overline{F_{B}^{\prime}}) 
		= A \cdot C \equiv (B + (A \centernot\implies C))
	\end{equation}
	\begin{equation}
		F(A, B, C) = F(A, B, 0) \equiv (\overline{C} + \overline{F_{C}^{\prime}})
		= \overline{A} \cdot \overline{B} \equiv (\overline{C} + \overline{A} 
		+ \overline{B})
	\end{equation}
	\begin{equation}
		F(A, B, C) = F(A, B, 1) \equiv (C + \overline{F_{C}^{\prime}})
		= (A \equiv B) \equiv (C + \overline{A} + \overline{B})
	\end{equation}
		
	\section{Все производные с помощью таблицы истинности}
	
	\begin{tabular}{|c|c|c|c|c|c|c|c|c|}
		\hline 
		A & 0 & 0 & 0 & 0 & 1 & 1 & 1 & 1 \\ 
		\hline 
		B & 0 & 0 & 1 & 1 & 0 & 0 & 1 & 1 \\ 
		\hline 
		C & 0 & 1 & 0 & 1 & 0 & 1 & 0 & 1 \\ 
		\hline 
		F & 1 & 1 & 0 & 0 & 0 & 0 & 0 & 1 \\ 
		\hline 
		F_{A}^{\prime} & 1 & 1 & 0 & 1 & 1 & 1 & 0 & 1 \\ 
		\hline 
		F_{B}^{\prime} & 1 & 1 & 1 & 1 & 0 & 1 & 0 & 1 \\ 
		\hline 
		F_{C}^{\prime} & 0 & 0 & 0 & 0 & 0 & 0 & 1 & 1 \\ 
		\hline 
		F_{AB}^{\prime\prime} & 1 & 0 & 1 & 0 & 1 & 0 & 1 & 0 \\ 
		\hline 
		F_{AC}^{\prime\prime} & 0 & 0 & 1 & 1 & 0 & 0 & 1 & 1 \\ 
		\hline 
		F_{BC}^{\prime\prime} & 0 & 0 & 0 & 0 & 1 & 1 & 1 & 1 \\ 
		\hline 
		F_{ABC}^{\prime\prime\prime} & 1 & 1 & 1 & 1 & 1 & 1 & 1 & 1 \\ 
		\hline 
	\end{tabular} 

	\section{Аналитический вид производных}
	
	Из задания 11:
	\begin{equation}
		F_{A}^{\prime} = B \implies C
	\end{equation}
	\begin{equation}
		F_{B}^{\prime} = A \implies C
	\end{equation}
	\begin{equation}
		F_{C}^{\prime} = A \cdot B
	\end{equation}
	Воспользовавшись таблицей истинности из предыдущего задания, найдём аналитический вид смешанных производных.
	
	\begin{equation}
		F_{AB}^{\prime\prime} = \overline{C}
	\end{equation}
	\begin{equation}
		F_{AC}^{\prime\prime} = B
	\end{equation}
	\begin{equation}
		F_{BC}^{\prime\prime} = A
	\end{equation}
	\begin{equation}
		F_{ABC}^{\prime\prime\prime} = 1
	\end{equation}

	\section{Разложение в ряд Маклорена в базисе $\{1, \oplus, \cdot\}$}
	Разложим функцию F в ряд Маклорена в базисе $\{1, \oplus, \cdot\}$.
	Получим полином Жегалкина.
	
	\begin{equation}
		F(A, B, C) = 1 \oplus A \oplus B \oplus A \cdot B 
		\oplus A \cdot B\cdot C
	\end{equation}
	
	\section{Разложение в ряд Тейлора в базисе $\{1, \oplus, \cdot\}$}
	Разложим функцию F в ряд Тейлора в каждой точке пространства в базисе
	$\{1, \oplus, \cdot\}$. При разложении будем использовать сокращенную запись 
	$\overline{A} = A \oplus 1$, $\overline{B} = B \oplus 1$, 
	$\overline{C} = C \oplus 1$. 
	
	\begin{equation}
	(0, 0, 0): F(A, B, C) = 1 \oplus A \oplus B \oplus A \cdot B 
	\oplus A \cdot B\cdot C
	\end{equation}
	
	\begin{equation}
	(0, 0, 1): F(A, B, C) = 1 \oplus A \oplus B \oplus 
	A \cdot B \cdot \overline{C}
	\end{equation}
	
	\begin{equation}
	(0, 1, 0): F(A, B, C) = \overline{B} \oplus A \cdot \overline{B}
	\oplus A \cdot C \oplus A \cdot \overline{B} \cdot C 
	\end{equation}
	
	\begin{equation}
	(0, 1, 1): F(A, B, C) = A \oplus \overline{B} \oplus A \cdot \overline{C} \oplus A \cdot \overline{B} \cdot \overline{C}
	\end{equation}
	
	\begin{equation}
	(1, 0, 0): F(A, B, C) = \overline{A} \oplus \overline{A} \cdot B
	 \oplus B \cdot C \oplus \overline{A} \cdot B \cdot C
	\end{equation}
	
	\begin{equation}
	(1, 0, 1): F(A, B, C) = \overline{A} \oplus B \oplus B \cdot \overline{C} 
	\oplus \overline{A} \cdot B \cdot \overline{C}
	\end{equation}
	
	\begin{equation}
	(1, 1, 0): F(A, B, C) = С \oplus \overline{A} \cdot \overline{B}
	 \oplus \overline{A} \cdot C \oplus \overline{B} \cdot C 
	 \oplus \overline{A} \cdot \overline{B} \cdot C
	\end{equation}
	
	\begin{equation}
	(1, 1, 1): F(A, B, C) = \overline{A} \oplus \overline{B} \oplus \overline{C} 
	\oplus \overline{A} \cdot \overline{C} \oplus \overline{B} \cdot \overline{C} \oplus \overline{A} \cdot \overline{B} \cdot \overline{C} 
	\end{equation}
	
	\section{Разложение в ряд Маклорена в базисе $\{0, \equiv, +\}$}
	Разложим функцию F в ряд Маклорена в базисе $\{0, \equiv, +\}$.
	
	\begin{equation}
	F(A, B, C) = A \equiv B \equiv C \equiv (A + C) 
	\equiv (B + C) \equiv (A + B + C)
	\end{equation}
	
	
	\section{Разложение в ряд Тейлора $\{0, \equiv, +\}$}
	Разложим функцию F в ряд Тейлора в каждой точке пространства в базисе 
	$\{0, \equiv, +\}$. При разложении будем использовать сокращённую запись: 
	$\overline{A} = A \equiv 0$, $\overline{B} = B \equiv 0$,
	$\overline{C} = C \equiv 0$,
	
	\begin{equation}
	(1, 1, 1): F(A, B, C) = A \equiv B \equiv C \equiv (A + C) 
	\equiv (B + C) \equiv (A + B + C)
	\end{equation}
	
	\begin{equation}
	(1, 1, 0): F(A, B, C) = 0 \equiv \overline{C} \equiv (A + B) 
	\equiv (A + \overline{C}) \equiv (B + \overline{C}) 
	\equiv (A + B + \overline{C})
	\end{equation}
	
	\begin{equation}
	(1, 0, 1): F(A, B, C) = 0 \equiv A \equiv \overline{B} 
	\equiv (\overline{B} + C) \equiv (A + \overline{B} + C)
	\end{equation}
	
	\begin{equation}
	(1, 0, 0): F(A, B, C) = 0 \equiv A \equiv (A + \overline{B}) 
	\equiv (\overline{B} + \overline{C}) 
	\equiv (A + \overline{B} + \overline{C})
	\end{equation}
	
	\begin{equation}
	(0, 1, 1): F(A, B, C) = 0 \equiv \overline{A} \equiv B \equiv 
	(\overline{A} + C) \equiv (\overline{A} + B + C)
	\end{equation}
	
	\begin{equation}
	(0, 1, 0): F(A, B, C) = 0 \equiv B \equiv (\overline{A} + B) 
	\equiv (\overline{A} + \overline{C}) 
	\equiv (\overline{A} + B + \overline{C})
	\end{equation}
	
	\begin{equation}
	(0, 0, 1): F(A, B, C) = \overline{A} \equiv \overline{B} 
	\equiv (\overline{A} + \overline{B} + C)
	\end{equation}
	
	\begin{equation}
	(0, 0, 0): F(A, B, C) = \overline{A} \equiv \overline{B} 
	\equiv (\overline{A} + \overline{B}) 
	\equiv (\overline{A} + \overline{B} + \overline{C})
	\end{equation}
\end{document}